\begin{appendices}

%
% The first appendix must be "Self-appraisal".
%
\chapter{Self-appraisal}

<This appendix should contain everything covered by the 'self-appraisal' criterion in the mark scheme. Although there is no length limit for this section, 2---4 pages will normally be sufficient. The format of this section is not prescribed, but you may like to organise your discussion into the following sections and subsections.>

\section{Critical self-evaluation}

\section{Personal reflection and lessons learned}

\section{Legal, social, ethical and professional issues}

<Refer to each of these issues in turn. If one or more is not relevant to your project, you should still explain {\em why} you think it was not relevant.>

\subsection{Legal issues}

\subsection{Social issues}

\subsection{Ethical issues}

\subsection{Professional issues}


%
% Any other appendices you wish to use should come after "Self-appraisal". You can have as many appendices as you like.
%
\chapter{External Material} \label{ExternalMaterial}
<This appendix should provide a brief record of materials used in the solution that are not the student's own work. Such materials might be pieces of codes made available from a research group/company or from the internet, datasets prepared by external users or any preliminary materials/drafts/notes provided by a supervisor. It should be clear what was used as ready-made components and what was developed as part of the project. This appendix should be included even if no external materials were used, in which case a statement to that effect is all that is required.>

Three types of external material were used in my solution: software libraries, an implementation of the ACES tone mapping operator, and 3D Models.

\section{Software Libraries} \label{SoftwareLibraries}

The libraries used in my renderer are listed below. For each one, a description is provided, along with a link. Note that the separation between my code and the external libraries is very clear in the code base, with all libraries being present under the \mintinline{bash}|/Application/Vendor| directory.

\vspace{20pt}

\noindent\begin{tabular}{|m{5em}|m{28em}|m{8em}|}
	\hline
	\textbf{Library Name} & \textbf{Description} & \textbf{Link} \\
	\hline\hline
	GLFW & A cross-platform utility library that provides windowing, OpenGL contexts and retrieves input events & \url{https://www.glfw.org/} \\
	\hline
	Glad & A library for loading pointers to the OpenGL functions & \url{https://glad.dav1d.de/} \\
	\hline
	GLM	& A maths library specifically designed for use with OpenGL & \url{https://github.com/g-truc/glm} \\
	\hline
	spdlog & A fast logging library & \url{https://github.com/gabime/spdlog} \\
	\hline
	stb\_image & An image loading library that is used when creating textures & \url{https://github.com/nothings/stb/blob/master/stb_image.h} \\
	\hline
	assimp & A library to import 3D models of various different file types & \url{https://github.com/assimp/assimp} \\
	\hline
\end{tabular}

\section{ACES Tone Mapping Operator} \label{ACESExternalMaterial}

The implementation of the ACES tone mapping operator, which is given in Listing \ref{ls:ACES}, is a slightly modified version of Stephen Hill's code. His code can be accessed via the following link: \url{https://github.com/TheRealMJP/BakingLab/blob/master/BakingLab/ACES.hlsl}

\section{3D Models}


%
% Other appendices can be added here following the same pattern as above.
%
\chapter{Mathematical Notation}

\begin{center}
	\begin{tabular}{ c c }
		\hline
		\begin{math}\vect{n}\end{math} & Normal vector \\
		\begin{math}\vect{l}\end{math} & Light direction \\
		\begin{math}\vect{v}\end{math} & View vector \\
		\begin{math}\vect{h}\end{math} & Half vector \\
		\begin{math}\vect{a}\cdot\vect{b}\end{math} & The dot product of vectors \begin{math}\vect{a}\end{math} and \begin{math}\vect{b}\end{math} \\
		\begin{math}\norm{\vect{a}}\end{math} & The norm of vector \begin{math}\vect{a}\end{math} \\
		\begin{math}\abs{x}\end{math} & The absolute value of \begin{math}x\end{math} \\
		\begin{math}x^+\end{math} & Clamp \begin{math}x\end{math} to \begin{math}0\end{math} if \begin{math}x<0\end{math} \\
		\begin{math}\mathcal{X}^+(x)\end{math} & Returns 1 if \begin{math}x > 0\end{math}, else returns 0 \\
		\begin{math}lerp(x, y, t)\end{math} & Linearly interpolates between \begin{math}x\end{math} and \begin{math}y\end{math} by the interpolant \begin{math}t\end{math}: \begin{math}lerp(x, y, t) = x(1 - t) + yt\end{math} \\
		\hline
	\end{tabular}
\end{center}

\chapter{Shader Code} \label{ShaderCode}

\section{Blinn-Phong Shader}

\subsection{Vertex Shader}

\subsection{Fragment Shader}

\section{Physically Based Shaders}

\subsection{PBS Vertex Shader}

\subsection{PBS Fragment Shader}

\subsection{Post Processing Vertex Shader}

\subsection{Post Processing Fragment Shader}

\end{appendices}
