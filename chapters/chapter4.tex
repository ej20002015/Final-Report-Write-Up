\chapter{Discussion}
\label{chapter4}

\section{Conclusions}

In conclusion, PBS models physical phenomena, such as the Fresnel effect and the influence of multiple incident lights, more accurately than Blinn-Phong shading. Consequently, Physically Based Shading renders frames that are more photorealistic than Blinn-Phong shading.

In the Introduction, when discussing the benefits of PBS, a distinction is made between the improvements to artist workflow and the enhanced photorealism. However, the investigation carried out in Chapter \ref{chapter3} showed that the two categories are not entirely independent, with some of the enhancements to photorealism also benefiting artists. Due to PBS modelling the Fresnel effect, it is simpler for artists to choose material properties, as these are not contingent on how an object is orientated relative to the viewer. Furthermore, PBS accurately models the impact multiple incident lights have on the appearance of an object, meaning artists have more freedom to place lights wherever they wish within a scene.

Overall, the use of physically based shading models in real-time rendering has been thoroughly explored, and a detailed analysis provided for why this technology has superseded Blinn-Phong shading.

\section{Future Work}

Looking to the future, the work carried out in this report could be expanded upon in a number of ways.

In Section \ref{DiffuseBRDFs}, the MERL BRDF database was briefly mentioned. Compiled by Matusik et al. at Mitsubishi Electric Research Laboratories, the database contains BRDF measurements for 100 different materials~\cite{MERL}. While developing the \textit{Disney Principled BRDF}, Burley uses this database as a ground truth to compare against various analytical BRDFs~\cite{Burley2012Physically}. This comparison is facilitated by constructing visualisations of the MERL and analytical BRDFs called "Image Slices". These Image Slices collapse the hyperdimensional BRDF functions into a more convenient 2D image that captures all of the important characteristics of a material's reflectance. Specifically, the Image Slices possess a number of distinct regions, with each exhibiting one of the following effects: the Fresnel reflectance, the specular response, the diffuse response, and \textit{retroreflection}. Retroreflection refers to cases where light strikes a material's surface and is preferentially reflected back along the incident direction. Typically this occurs in materials with very rough microgeometry, and is one of the key effects that Oren and Nayar seek to model in their BRDF~\cite{OrenAndNayar}. Combining the MERL BRDF database with these Image Slices represents a robust means of evaluating shading models. In the future, such a combination could be used to compare the Blinn-Phong shading model to multiple different physically based shading models. Not only does this have the potential to uncover further insights into how the physically based models are more realistic than Blinn-Phong, but it would also provide a comprehensive analysis of how all the physically based shading models compare to one another.

In the Introduction, there is a brief discussion on the benefits that are afforded to artist workflow by adopting PBS. Looking into the relevant literature, it seems there is a dearth of knowledge in this particular area - there are no papers that perform an investigative, scientific study into the impact that different shading models have on artist workflow. The minimal amount of information that does exist is anecdotal in nature, and attests to the same points that are covered in that previous discussion. These anecdotal observations are from respected academics within the real-time graphics industry, so their comments should by no means be brushed aside. However, a scientific, systematic and thorough study into the relationship between shading model and artist workflow, would likely provide a wealth of new insights. If this study compared physically based shading models to the Blinn-Phong shading model, then it would form a good companion to the information laid out in this report. A prominent part of such a study would be energy conservation. Unlike the physically based shading model implemented in Section \ref{BRDFChoiceAndImplementation}, the proportions of the diffuse and specular terms in the Blinn-Phong model are unconstrained, and do not have to add to 1. Studying Equation \ref{eq:BlinnPhong}, it is clear that these proportions are controlled by artists. Consequently, when artists use the Blinn-Phong model, it is much easier to violate conservation of energy than to comply with it. The study would likely have to be carried out by getting experienced artists to participate in user research. Perhaps the Blinn-Phong and physically based renderers that were developed in this report could be utilised.

As mentioned in Section \ref{PBRLighting}, the physically based shading model implemented in this report only accounts for indirect illumination using a crude ambient term. When adopting PBS, many of the enhancements to photorealism are obtained by using a more sophisticated method for computing indirect lighting. These enhancements could not be covered in this report. Naturally, a future piece of work could iterate on this report's implementation to add Image Based Lighting and Irradiance Mapping to the shading model. A comparison could then be performed to highlight how these improvements contribute to rendered frames that are more photorealistic than Blinn-Phong shading.